% Feldspar-EF.tex
\begin{hcarentry}{Feldspar}
\label{feldspar}
\report{Emil Axelsson}%11/12
\status{active development}
\makeheader

Feldspar is a domain-specific language for digital signal processing (DSP). The
language is embedded in Haskell and developed in co-operation by Ericsson,
Chalmers University of Technology (G\"oteborg, Sweden) and E\"otv\"os Lor\'and
(ELTE) University (Budapest, Hungary).

The motivating application of Feldspar is telecoms processing, but the language
is intended to be useful for DSP in general. The aim is to allow DSP functions
to be written in pure functional style in order to raise the abstraction level of
the code and to enable more high-level optimizations. The current version consists
of an extensive library of numeric and array processing operations as well as a
code generator producing C code for running on embedded targets.

The current version deals with the data-intensive numeric algorithms which are
at the core of any DSP application. We have recently added support for the expression
and compilation of parallel algorithms. As future work remains to extend the language
to deal with interaction with the environment (e.g., processing of streaming data) and
to support compilation to heterogeneous multi-core targets.

\FurtherReading
\begin{compactitem}
\item \url{https://github.com/Feldspar/feldspar-language}
\item \url{http://hackage.haskell.org/package/feldspar-language}
\item \url{http://hackage.haskell.org/package/feldspar-compiler}
\end{compactitem}
\end{hcarentry}
