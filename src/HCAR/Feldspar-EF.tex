% Feldspar-EF.tex
\begin{hcarentry}{Feldspar}
\label{feldspar}
\report{Emil Axelsson}%11/14
\status{active development}
\makeheader

Feldspar is a domain-specific language for digital signal processing (DSP). The
language is embedded in Haskell and is currently being developed by projects at
Chalmers University of Technology~\cref{chalmers}, SICS Swedish ICT AB and
Ericsson AB.

The motivating application of Feldspar is telecoms processing, but the language
is intended to be useful for DSP and numeric code in general. The aim is to
allow functions to be written in pure functional style in order to raise the
abstraction level of the code and to enable more high-level optimizations. The
current version consists of a library of numeric and array processing operations
as well as a code generator producing C code for running on embedded targets.

The official packages \href{http://hackage.haskell.org/package/feldspar-language}{feldspar-language} and \href{http://hackage.haskell.org/package/feldspar-compiler}{feldspar-compiler} contain the language for pure computations and its C back end, respectively.

Additionally, \href{https://github.com/emilaxelsson/feldspar-io}{feldspar-io} (not yet released, but fully usable) adds an ``IO-like'' monad for making interactive Feldspar programs and binding to external C libraries. Ongoing work involves using \pre|feldspar-io| to implement more high-level libraries for streaming and interactive programs. Two examples of such libraries are:
\begin{compactitem}
\item \href{https://github.com/emilaxelsson/feldspar-synch}{feldspar-synch} -- a synchronous data-flow library
\item \href{https://github.com/koengit/zeldspar}{zeldspar} -- a Ziria-like EDSL
\end{compactitem}

\FurtherReading
\begin{compactitem}
\item Official home page: \url{http://feldspar.github.io}
\item \href{http://www.cse.chalmers.se/~emax/documents/persson2015programmable.pdf}{Recent paper (TFP 2015)} about controlling the signatures of generated C functions
\end{compactitem}
\end{hcarentry}
