% Feldspar-EF.tex
\begin{hcarentry}{Feldspar}
\label{feldspar}
\report{Emil Axelsson}%11/16
\status{active development}
\makeheader

Feldspar is a domain-specific language for digital signal processing (DSP).
The language is embedded in Haskell and has been developed as a collaboration,
in different phases, between Chalmers University of
Technology, ELTE University, SICS Swedish ICT AB and Ericsson AB.

The motivating application of Feldspar is telecoms processing, but the
language is intended to be useful for DSP and numeric code in general as well
as for programming embedded systems. The aim is to allow functions to be
written in functional style in order to raise the abstraction level of the
code and to enable more high-level optimizations.

The currently recommended Feldspar implementation is
\mbox{\href{http://hackage.haskell.org/package/raw-feldspar}{RAW-Feldspar}}.
Its
\mbox{\href{https://github.com/Feldspar/raw-feldspar/blob/master/README.md}{README}}
file is a good starting point for getting to know Feldspar.

RAW-Feldspar provides libraries for numeric and array processing operations,
and supports file handling, calls to external C libraries, concurrency, etc.
It also comes with a code generator producing C code for running on embedded
targets.

For reference, the original Feldspar implementation is available in the
packages

\begin{compactitem}
\item
  \href{http://hackage.haskell.org/package/feldspar-language}{feldspar-language}
  -- language front end
\item
  \href{http://hackage.haskell.org/package/feldspar-compiler}{feldspar-compiler}
  -- C back end
\end{compactitem}

Ongoing work involves using RAW-Feldspar to implement more high-level
libraries for streaming and interactive programs. Two examples of such
libraries are:

\begin{compactitem}
\item \href{https://github.com/koengit/zeldspar}{zeldspar} -- a Ziria-like
  EDSL
\item \href{https://github.com/emilaxelsson/feldspar-synch}{feldspar-synch} --
  a synchronous data-flow library
\end{compactitem}

\href{https://github.com/kmate/raw-feldspar-mcs}{raw-feldspar-mcs} is a
library built on top of RAW-Feldspar that generates code for running on NUMA
architectures such as the \href{http://www.parallella.org}{Parallella}.

There is also ongoing work to support hardware/software co-design in
RAW-Feldspar.

\FurtherReading
\begin{compactitem}
\item Official home page: \url{http://feldspar.github.io}
\end{compactitem}
\end{hcarentry}
