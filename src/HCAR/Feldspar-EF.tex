% Feldspar-EF.tex
\begin{hcarentry}[updated]{Feldspar}
\label{feldspar}
\report{Emil Axelsson}%11/15
\status{active development}
\makeheader

Feldspar is a domain-specific language for digital signal processing (DSP). The
language is embedded in Haskell and is currently being developed by projects at
Chalmers University of Technology~\cref{chalmers}, SICS Swedish ICT AB and
Ericsson AB.

The motivating application of Feldspar is telecoms processing, but the language
is intended to be useful for DSP and numeric code in general. The aim is to
allow functions to be written in pure functional style in order to raise the
abstraction level of the code and to enable more high-level optimizations. The
current version consists of a library of numeric and array processing operations
as well as a code generator producing C code for running on embedded targets.

The official packages
\href{http://hackage.haskell.org/package/feldspar-language}{feldspar-language}
and
\href{http://hackage.haskell.org/package/feldspar-compiler}{feldspar-compiler}
contain the language for pure computations and its C back end, respectively.

Additionally, we are working on a completely new implementation of Feldspar,
\href{https://github.com/emilaxelsson/raw-feldspar}{RAW-Feldspar} (not yet
released, but fully usable). This implementation uses a slightly different
language design that gives better control over things like memory allocation. It
also extends Feldspar with a monad that supports interaction with the operating
system, calls to external C libraries, concurrency, etc.

Ongoing work involves using RAW-Feldspar to implement more high-level libraries
for streaming and interactive programs. Two examples of such libraries are:

\begin{compactitem}
\item \href{https://github.com/koengit/zeldspar}{zeldspar} -- a Ziria-like EDSL
\item \href{https://github.com/emilaxelsson/feldspar-synch}{feldspar-synch} -- a synchronous data-flow library
\end{compactitem}

\href{https://github.com/kmate/raw-feldspar-mcs}{raw-feldspar-mcs} is a
library built on top of RAW-Feldspar that generates code for running on NUMA
architectures such as the \href{http://www.parallella.org}{Parallella}.

There is also ongoing work to generate VHDL from RAW-Feldspar programs.

\FurtherReading
\begin{compactitem}
\item Official home page: \url{http://feldspar.github.io}
\end{compactitem}
\end{hcarentry}
