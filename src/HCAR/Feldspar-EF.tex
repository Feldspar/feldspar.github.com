% Feldspar-EF.tex
\begin{hcarentry}{Feldspar}
\label{feldspar}
\report{Emil Axelsson}%11/13
\status{active development}
\makeheader

Feldspar is a domain-specific language for digital signal processing (DSP). The
language is embedded in Haskell and is currently being developed by projects at
Chalmers University of Technology~\cref{chalmers}, SICS Swedish ICT AB and
Ericsson AB.

The motivating application of Feldspar is telecoms processing, but the language
is intended to be useful for DSP in general. The aim is to allow DSP functions
to be written in pure functional style in order to raise the abstraction level of
the code and to enable more high-level optimizations. The current version consists
of an extensive library of numeric and array processing operations as well as a
code generator producing C code for running on embedded targets.

At present, Feldspar can express the pure data-intensive numeric algorithms
which are at the core of any DSP application. There is also support for the
expression and compilation of parallel algorithms.

Ongoing work, presented at \href{http://ifl2014.github.io/submissions/ifl2014_submission_23.pdf}{IFL 2014},
extends Feldspar with basic input/output capabilities and adds a library to
express streaming systems using a synchronous programming model. Future work
involvs extending and improving the system programming part of the language, and
adding support for compilation to heterogeneous multi-core targets.

\FurtherReading
\begin{compactitem}
\item \url{http://feldspar.github.io}
\item \url{http://hackage.haskell.org/package/feldspar-language}
\item \url{http://hackage.haskell.org/package/feldspar-compiler}
\end{compactitem}
\end{hcarentry}
